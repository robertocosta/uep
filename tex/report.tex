\documentclass[italian, 11pt]{article}
\usepackage{float}
%%% font & looks %%%
\usepackage{mathpazo}
%\usepackage{palatino}
%\usepackage{kpfonts}
%\usepackage{times}
%\usepackage{charter}
%\usepackage{utopia}

% list of fonts available here - http://www.tug.dk/FontCatalogue/
% and here - https://www.sharelatex.com/learn/Font_typefaces

\usepackage{microtype}
\usepackage[utf8x]{inputenc}
\usepackage[T1]{fontenc}
\usepackage[parfill]{parskip}
\usepackage[a4paper, left=25mm, top=25mm, right=25mm, bottom=25mm]{geometry}
\usepackage{bm}
\usepackage[labelfont=bf,font={small,it}]{caption}
\usepackage[usenames,dvipsnames,svgnames,table]{xcolor}

%%% boxes %%%
\usepackage{framed}
%\setlength\FrameSep{0.5em}

% math
\usepackage{amsmath}
\usepackage{amssymb}
\usepackage{amsthm} 
\usepackage{wasysym}

% graphics
\usepackage{multirow}
\usepackage{placeins}
\usepackage{graphicx}
\usepackage{epstopdf}
\usepackage{pgfplots}
  \pgfplotsset{compat=newest}
  %% the following commands are needed for some matlab2tikz features
  \usetikzlibrary{plotmarks}
  \usetikzlibrary{arrows.meta}
  \usepgfplotslibrary{patchplots}
  \usepackage{grffile}
  \usepackage{amsmath}
  
  
% extra
\usepackage{todonotes} %\todo{ note } , \listoftodos
\usepackage{blindtext}
\usepackage{lipsum}
\usepackage{tikz}
\usetikzlibrary{patterns}
\usetikzlibrary{decorations.pathreplacing,angles,quotes}
\usepackage{chemformula}
\usepackage{textcomp}

% better tables
\usepackage{booktabs}

%%% links %%%
\usepackage{hyperref}
\hypersetup{
	linktocpage,
	colorlinks,
    citecolor=black,
    filecolor=black,
    linkcolor=black,
    urlcolor=black
}

\usepackage[framed,numbered]{matlab-prettifier}
\tikzset{
  font={\fontsize{9pt}{12}\selectfont}}

%%% references %%%
\usepackage[noabbrev,capitalize,nameinlink]{cleveref}

% new commands
\newcommand{\horline}{\rule{1\linewidth}{0.9pt}}
\newcommand{\curl}{\nabla \times}
\renewcommand{\div}{\nabla \cdot}
\newcommand{\laplac}{\nabla^2}
\newcommand{\grad}{\nabla}

\newtheorem{p}{\\[5mm] \large Problem}
\newenvironment{s}{%\small%
        \begin{trivlist} \item \textbf{Solution}. }{%
            \end{trivlist}}%
\begin{document}
\begin{center}
	%\horline\\
	\Large Wireless Systems And Networks\\
	\huge \textbf{Project: Unequal Error Protection}\\[3mm]
	\begin{framed}
		\Large \textbf{Workgroup} \\[2mm]
		\normalsize \textbf{Costa} Roberto -- \textbf{Zanol} Riccardo
	\end{framed}
	%\horline
\end{center}	
\FloatBarrier
\section{Sommario}
La tesina si prefigge di implementare un protocollo per trasmettere flussi video attraverso una rete non completamente affidabile, regolando la ridondanza aggiunta ai dati in base alla loro importanza e in base alle necessità di ritrasmissione dei ricevitori.
Verrà analizzata la correttezza della trasmissione (in termini di Bit Error Rate e PSNR) al variare della Bit Error Rate del canale. Verranno inoltre analizzate la complessità computazionale di codifica e decodifica e il tempo ad esse associato, al variare della ridondanza aggiunta.\\
Il progetto è stato realizzato in C++; Python è stato usato per riportare graficamente i risultati ottenuti.
\section{Introduzione}
A livello applicazione, il protocollo di codifica video usato è H.264/SVC (Scalable Video Coding), in cui il flusso video è codificato in n diversi flussi con differente priorità.\\
Il flusso video viene inizialmente segmentato in una serie di GOP (gruppi di immagini); successivamente, per ogni segmento, il flusso a priorità maggiore (layer 0) manda l'informazione relativa l'immagine iniziale e alla variazione delle basse frequenze della trasformata discreta di Fourier di ogni frame successivo; tale layer è indispensabile per la decodifica, mentre i flussi con priorità minore aggiungono qualità al video decodificato, ma non sono indispensabili per la decodifica.\\
Per questo motivo il layer 0 necessita di una maggiore ridondanza, quando viene trasmesso attraverso il canale fisico.\\



\section{Approccio tecnico}
\subsection{Obbiettivi}
\subsection{Diagramma}
\subsection{Modelli matematici}
i.i.d. channel w. Error rate\\

\section{Risultati}

\section{Conclusioni}

\section{Bibliografia}
%\begin{figure}[H]
%	\makebox[\textwidth][c]{\includegraphics[width=16cm]{G10labE_f3.eps}}
%	\caption{Componenti di $\hat{s}$}\label{fig:s2}
%\end{figure}

\end{document}
